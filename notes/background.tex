
\section{Background}
\label{sec:context}

We review techniques that are used in our algorithmic proposals and
computational experiments.

\subsubsection{Stability}

Many statistical quantities are based on the idea that meaningful conclusions
should be stable to perturbations \cite{yu2013stability}. In this work, we make
use of stability selection, a technique for evaluating significance in
high-dimensional linear models \cite{meinshausen2010stability}.

For a dataset $\mathbf{X}\in \reals^{n \times p}$ and $\mathbf{y} \in
\reals^{n}$, stability selection proceeds as follows. First, $B$ subsamples of
size $\lfloor \frac{n}{2} \rfloor$ are drawn from the data, and for each a lasso
regression is run over a grid of regularization parameters $\lambda$. Each of
these regressions results in $p$ coefficient trajectories
$\hat{\beta}_{j}\left(\lambda\right)$, and important features are expected to
become nonzero earlier on the regularization path (that is, even with large
regularization $\lambda$). For a given $\lambda$ and feature $j$, let
$\hat{\Pi}_{j}\left(\lambda\right)$ measure the fraction of subsamples for which
$\absarg{\hat{\beta}_j\left(\lambda\right)} > 0$. The paths
$\left(\hat{\Pi}_{j}\left(\lambda\right)\right)_{j = 1}^{p}$ describe the
importance of each of the $p$ regression features. For a given stringency
threshold $\pi_{thr}$, the procedure selects features $\hat{S} = \{j :
\max_{\lambda} \hat{\Pi}_{j}\left(\lambda\right) \geq \pi_{thr}\}$.

Let $q_{\Lambda} = \Eabsarg{\hat{S}_{\Lambda}}$ denote the expected number of
selected features. It can be shown that, for $\pi_{thr} \geq \frac{1}{2}$, and
assuming a sufficiently well-behaved $\hat{S}$, the expected number of falsely
selected features is bounded by $\frac{1}{2\pi_{thr} - 1}
\frac{q_{\Lambda}^2}{p}$. The term $\frac{q_{\Lambda}}{p}$ is like the typical
fraction of selected features; the coefficient $\frac{q_{\Lambda}}{2\pi_{thr} -
  1}$ describes the fraction of those that are expected to be false positives.

\subsubsection{Feature Learning}

In this work, we consider three particular feature learning algorithms. The
first is called the Variational Autoencoder (VAE). Like principal components
analysis, this algorithm learns an $L$-dimensional reduction of a dataset by
optimizing a reconstruction objective. Formally, the algorithm first posits a
generative model $p\left(z\right)p_{\xi}\left(x \vert z\right)$ of the data;
$p\left(z\right)$ is a prior on latent features and $p_{\xi}\left(x \vert
x\right)$ is a likelihood parameterized by $\xi$. The algorithm finds a pair
$\xi, \varphi$ maximizing the lower bound,

\begin{align*}
\log p_{\xi}\left(x\right) \geq  \mathbb{E}_{q_{\varphi}}\left[\log p]_{\xi}(x \mid z)\right]-D_{KL}\left(q_{\varphi}(z \mid x) \| p(z)\right)
\end{align*}

where $q_{\varphi}\left(z \vert x\right) = N\left(\mu_{\varphi}\left(x\right),
\sigma^{2}_{\varphi}\left(x\right)\right)$ maps raw data examples to
distributions in a latent space of more meaningful features. This optimization
problem is nonconvex, and is typically solved through a variant of stochastic
gradient descent. There are many particular implementations of VAEs; our
experiments are based on \citep{van2017neural}.

Second, we investigate features learned in a supervised way through a
Convolutional Neural Network (CNN). For regression, a CNN optimizes the
empirical estimate of the risk $\mathbf{E}\|y -
f_{W_{1:J}}\left(x\right)^{T}\beta\|_{2}^{2}$ over $W_{1:J}$ and $\beta$. Here,
$f_{W_{1:J}}$ transforms the raw input into the ``last layer''’s features, and
is defined recursively according to

\begin{align*}
f^{j}_{W_{1:j}}\left(x\right) &= \sigma\left(W_{j}f^{j - 1}_{W_{1:(j - 1)}}\left(x\right)\right)\\
f^{0}\left(x\right) &= x
\end{align*}

for $\sigma\left(x\right)$ defined as $\sigma\left(x\right) := x \indic{x \geq
  0}$ and matrices $W$ restricted to the set of convolution operators. Like in
the VAE, this optimization is nonconvex and is typically found through
first-order optimization methods. Our particular implementation is the CBR
architecture described in \citep{raghu2017svcca}.

Third, we consider a random convolutional features (RCF) model
\citep{rahimi2008weighted}. Here, a random sample of $L$ training examples
$x_{i_1}, \dots, x_{i_L} \in \reals^{w \times h \times c}$ are selected; the
$x_{i}$'s are assumed to be $c$-channel images with dimension $w\times h$. For
each sample, a random $s \times s$ patch, which we call $w_{p} \in \reals^{s
  \times s \times c}$, is extracted. For any $c$-channel image $x$, the $l^{th}$
feature $z_{l}$ is found by convolving $x$ with $w_{l}$ and spatially averaging
over all activations.

To train an RCF, the training data are first featurized into $\mathbf{Z} \in
\reals^{n \times L}$ using random image patches, as described above. Then, a
ridge regression model is trained, solving
\begin{align*}
\hat{\beta} := \arg \min_{\beta} \|y - \mathbf{Z}\beta\|_{2}^{2} + \lambda \|\beta\|_{2}
\end{align*}
For a new example $x^{\ast}$, the same image patches $w_{1}, \dots, w_{L}$ are
used to create a featurization $z^{\ast}$, and predictions are made with
$z^{\ast T}\hat{\beta}$. Unlike either the VAE or CNN, this model does not
require gradient based training, and it can serve as a fast, and often
effective, baseline.

\subsubsection{Procrustes Analysis}
Procrustes Analysis gives a way of aligning multiple tables. Given two centered
tables $\mathbf{X}$ and $\mathbf{Y}$, the Procrustes problem finds the rotation
matrix $\mathbf{R}$ solving the optimization,

\begin{align*}
\min_{\mathbf{R} \in \mathcal{O}\left(p, p\right)} \|\mathbf{X} - \mathbf{Y}\mathbf{R}\|^{2}_{F},
\end{align*}

where $\mathcal{O}\left(p, p\right)$ denotes the space of $p\times p$
orthonormal matrices. Using the associated Lagrangian,

\begin{align*}
\max_{\mathbf{R}, \mathbf{\Lambda} \succeq 0} \|\mathbf{X} - \mathbf{Y}\mathbf{R}\|_{F}^{2} - \mathbf{\Lambda}\left(\mathbf{R}^{T}\mathbf{R} - \mathbf{I}\right)
\end{align*}

the solution can be shown to be $\hat{\mathbf{R}} = \mathbf{U}^{T}\mathbf{V}$
for $\mathbf{U}$ and $\mathbf{V}$ obtained by the SVD of
$\mathbf{X}^{T}\mathbf{Y}$. For $B$ matrices $\mathbf{X}_{1}, \dots,
\mathbf{X}_{B}$, the generalized Procrustes problem finds $B$ rotations
$\mathbf{R}_{1}, \dots, \mathbf{R}_{B}$ and mean matrix $\mathbf{M}$ solving

\begin{align*}
\min_{\mathbf{R}_{1}, \dots, \mathbf{R}_{B} \in \mathcal{O}\left(p, p\right), M} \sum_{b = 1}^{B} \|\mathbf{X}_{b}\mathbf{R}_{b} - \mathbf{M}\|_{F}^{2}.
\end{align*}

While there is no closed form solution, the optimization can be solved by
cyclically updating each $\mathbf{R}_{b}$ via standard Procrustes problems and
then updating $\mathbf{M} = \frac{1}{B} \sum_{b = 1}^{B} \mathbf{X}_{b}
\mathbf{R}_{b}$.

\subsubsection{Representation Analysis}

Our approach is closely related to Singular Vector CCA (SVCCA), a method used to
compare features learned across deep learning models. In SVCCA, the two
representations are identified with two matrices, $X \in \reals^{n \times
  l_{1}}$ and $Y \in \reals^{n \times l_{2}}$. The $ij^{th}$ cell in each matrix
corresponds to the activation of neuron $j$ on sample $i$. A representation is a
pattern of activations across a collection of neurons.

To compare representations, SVCCA first computes singular value decompositions
$X = U_{X}D_{X}V_{X}^{T}$ and $Y = U_{Y}D_{Y}V_{T}^{T}$. The coordinates of
these matrices with respect to the top $K$ singular vector directions are then
extracted, $Z_{X} = U_{X, 1:K}D_{X, 1:K}$ and $Z_{Y} = U_{Y, 1:K}D_{Y, 1:K}$.
Finally, a canonical correlation analysis is performed on these coordinate
matrices. That is, the top CCA directions $u_{1}, v_{1}$ are found by optimizing

\begin{align*}
  \text{maximize }& u^{T}\hat{\Sigma}_{Z_{X}}^{-\frac{1}{2}}\hat{\Sigma}_{Z_{X}Z_{Y}}\hat{\Sigma}_{Z_{Y}}^{-\frac{1}{2}}v \\
  \text{subject to } & \|u\|_{2} = \|v\| = 1
\end{align*}

and subsequent directions are found by solving the same problem after
orthogonalizing $Z_{X}$ and $Z_{Y}$ to previously extracted directions. The
value of the objective for each of the $K$ directions extracted is denoted
$\rho_{k}$, and the overall similarity between the two representations is taken
to be the average of these canonical correlations: $\frac{1}{K}\sum_{k = 1}^{K}
\rho_{k}$.

Note that, while in principle, it would be possible to perform a CCA on the
activations $X$ and $Y$ directly, for representations with many neurons, the
dimensionality reduction step can simplify computation, because it avoids
inverting a high-dimensional covariance matrix.

\subsection{Sparse Components Analysis}

Like SVCCA, we reduce dimensionality of our learned features before comparing
them. We use both PCA and a variant called Sparse Components Analysis (SCA)
\citep{chen2020new}. For a given data matrix $X$ and a number of components $K$,
and sparsity parameter $\gamma$, SCA solves the optimization

\begin{align*}
  \text{maximize }_{Z, B, Y} &\|X - Z B Y^{T}\|_{F} \\
  \text{subject to }Z \in &\mathcal{O}\left(n, K\right) \\
  Y \in &\mathcal{O}\left(p, k\right) \\
  \|Y\|_{1} &\leq \gamma.
\end{align*}

The matrix $Y$ provides the $K$ SCA loadings, and the product $ZB$ provide the
coordinates of each sample with respect to that basis. Note that if $B$ is
forced to be diagonal, then this optimization reduces to sparse PCA. This
optimization does not directly provide an ordering of the loadings. However, the
proportion of variance explained by each dimension can still be computed, and
this can be used to order the dimensions.
