
\section{Simulation}

We use a simulation experiment to understand properties of the proposed
algorithms. Our key questions are,

\begin{enumerate}
\item How different are the features learned by supervised and unsupervised
  feature learning algorithms $T\left(\cdot; \theta\right)$?
\item Does the choice of dimensionality reduction used in $\mathcal{A}$ affect
  downstream inferences?
\item When we vary the relative sizes of $I$ and $I^{C}$, we expect a trade-off
  between feature learning and inferential quality. What is the nature of that
  trade-off?
\item Is it ever possible to substitute the RCF feature learner for either of
  the more time-consuming VAE or CNN models?
\end{enumerate}

To answer these questions, we evaluate our proposal using a full factorial with
three factors,

\begin{enumerate}
\item Relative sizes of $I$ and $I^{C}$: We use $\absarg{I} \in \left\{0.15n, 0.5n, 0.8n\right\}$.
\item Dimensionality reduction procedure: We use both PCA and sparse PCA.
\item Algorithm used: We train CNN, VAE and RCF models.
\end{enumerate}

\subsection{Simulated Data}

The key ingredient in this simulation is the construction of a dataset where all
``true'' features are directly controlled. To motivate the simulation, imagine
that we are examining a set of tumor pathology slides, with the hope of
recovering features that are predictive of survival time. Each pathology slide
gives a view into a cellular ecosystem — there are several types of cells, with
different sizes, density, and affinities to one another.

In the simulation, we first generate these underlying properties of each slide.
Survival is defined as a linear function of these underlying properties. Then,
images are generated that reflect these properties. The essential characteristic
of this simulation is that the analyst only has access to the images and
survival times, not the true predictors of the linear model. A good feature
extractor $T\left(x; \hat{\theta}\left(\mathcal{D}\right)\right)$ should recover
these important cell ecosystem properties from the images alone. Example images
for varying values of $y$ are given in Figure \ref{fig:matern_example}.

We now give details. Our images are simulated as two-dimensional marked Log Cox
Matern Process (LCMP) \cite{diggle2013}. Recall that a Matern process is a
Gaussian process with covariance function

\begin{align*}
C_{\nu, \alpha}(\|x - y\|)=\sigma^{2} \frac{2^{1-\nu}}{\Gamma(\nu)}\left(\sqrt{2 \nu} \frac{\|x - y\|}{\alpha}\right)^{\nu} K_{\nu}\left(\sqrt{2 \nu} \frac{\|x - y\|}{\alpha}\right).
\end{align*}

$\alpha$ acts like a bandwidth parameter and $\nu$ controls the roughness of the
simulated process. A LCMP process with $K$ classes is constructed as follows.
First, a nonnegative process $\Lambda\left(x\right)$ is simulated along the
image grid, $\Lambda\left(x\right) \exp{\mathcal{N}\left(0,
  \mathbf{C}_{\nu_{\Lambda}, \alpha_{\Lambda}}\right)}$, where
$\mathbf{C}_{\nu_{\Lambda}, \alpha_{\Lambda}}$ is the covariance matrix induced
by $C_{\nu_{\Lambda}, \alpha_{\Lambda}}$. This represents a baseline intensity
that determines the location of cells, regardless of cell type. Then, $K$
further processes are simulated, $B_{k}\left(x\right) \exp{b_{k} +
  \mathcal{N}\left(0, \mathbf{C}_{\nu_{B}, \alpha_{B}}\right)} $. These
processes will reflect the relative frequencies of the $k$ classes at any
location $x$; the intercept $b_k$ makes a class either more or less frequent
across all positions $x$.

Given these intensity functions, we can simulate $N$ cell locations by drawing
from an inhomogeneous Poisson process with intensity $\Lambda\left(x\right)$.
For a cell at location $x$, we assign it cell type $k$ with probability
$\frac{B_{k}^{\tau}\left(x\right)}{\sum_{k^\prime = 1}^{K}
  B^{\tau}_{k^\prime}\left(x\right)}$. Here, we have introduced a temperature
parameter $\tau$ which controls the degree of mixedness between cell types at a
given location.

To complete the procedure for simulating images, we add two last source of
variation — cell size. The cells from class $k$ are drawn with a random radius
drawn from $\text{Gamma}\left(5, \lambda_{k}\right)$. A summary of all
parameters used to generate each image is given in Table \ref{tab:sim_params}.
These parameters are the ``true'' underlying features associated with the
observed images; they give the most concise description of the variation
observed across the images.

\begin{figure}
  \centering
  \includegraphics[width=0.9\textwidth]{generation_mechanism}
  \caption{Example simulated images, for low (top), average (middle), and high
    (bottom) values of $y$. For each sample, three relative intensity functions
    are generated, shown as a greyscale heatmap (left three displays for each
    group). Samples drawn from each process are overlaid as circles. The final
    images (display at the right for each group) overlay the points from the
    three processes, discarding the original generating intensity function.
    Evidently, small $y$ values are associated with smoother, less structured
    intensity functions.}
  \label{fig:matern_example}
\end{figure}

\begin{table}[]
\begin{tabular}{|p{0.1\linewidth}|p{0.4\linewidth}|p{0.16\linewidth}|p{0.15\linewidth}|}
\hline
\textbf{Feature}              & \textbf{Description}                                 & \textbf{Influence}          & \textbf{Range}             \\
\hline
$N_i$                & The total number of cells.                                    & 0.5                         & $\left[50, 1000\right]$    \\
\hline
$\nu_{i,\Lambda}$    & The roughness of the underlying intensity process.            & -0.5                        & $\left[0, 8\right]$        \\
\hline
$\alpha_{i,\Lambda}$ & The roughness of the underlying intensity process.            & -0.5                        & $\left[0, 8\right]$        \\
\hline
$b_{ik}$             & The intercept controlling the overall frequency of class $k$. & 1 for $k = 1$,\ -1 otherwise & $\left[-0.15, 0.15\right]$ \\
\hline
$\nu_{iB}$           & The roughness of the relative intensity processes.            & -0.5                        & $\left[0, 3\right]$        \\
\hline
$\alpha_{iB}$        & The bandwidth of relative intensity processes.                & -0.5                        & $\left[0, 3\right]$        \\
\hline
$\tau_{i}$           & The temperature used in cell type assignment.                 & 0.5                         & $\left[0, 3\right]$        \\
\hline
$\lambda_{ik}$       & The shape parameter controlling the sizes of each cell type.  & 1 for $k = 1$,\ 0 otherwise  & $\left[100, 500\right]$   \\
\hline
\end{tabular}
\caption{Our simulation mechanism is governed by the above parameters.
  Parameters $N_i$ and $\lambda_{ik}$ control the number and sizes of imaginary
  cells. $\nu_{i, \Lambda}$, $\alpha_{i, \Lambda}$, $b_{ik}$, $\nu_{iB}$,
  $\alpha_{iB}$, and $\tau_{i}$ control the overall and relative intensities of
  the Matern process from which the cell positions are drawn. Example draws are
  displayed in Figure \ref{fig:matern_example}}
\label{tab:sim_params}
\end{table}


These features are the underlying properties used to generate survival time.
Specifically, we generate a survival time using $y_i = \sum_{k}
\text{Influence}\text{Feature}_{ik}$ for the values Influence given in Table
\ref{tab:sim_params}. Note that there is no additional noise: if the
$\text{Feature}_{ik}$'s were known for each sample, then the $y$'s could be
predicted perfectly. Therefore, the simulation gauges the capacity of the
feature learners to recover these known underlying features. These features are
themselves all drawn from uniform distributions between the ranges given in the
table.

\subsection{Distributed features}

Figure \ref{fig:distributed_hm} shows the activations of learned features across
2000 samples for two bootstrap resamples from each of the feature learning
algorithms when using 90\% of the training data for feature learning. In the
three cases, the learned features correspond to

\begin{itemize}
\item CNN: Activations from the last layer of neurons, used directly as input
  for the regression. There are a total of 512 nonnegative
  features\footnote{They are nonnegative because they follow an ReLU layer.}.
\item VAE: Spatially-pooled activations from the middle, encoding layer of the
  variational autoencoder. There are a total of 64 real-valued features.
\item RCF: The spatially-pooled activations corresponding to each of 1048 random
  convolutional features.
\end{itemize}

Our first observation is that there is no simple correspondence between learned
and source features (i.e., parameters of the underlying simulation). For
example, it is not the case that one set of features represents the number of
cells $N$, and a different set maps to the roughness $\alpha, \nu$. Rather,
there appear to be clusters of learned features, and each cluster corresponds to
a pattern of correlation across multiple source features. We also find large
subsets of features, across all models, that are only weakly correlated with any
source feature.

Next, note that certain source features are ``easier'' to represent than others,
in the sense that more of the learned features are strongly correlated with
them. Many features are correlated with $N_{i}$, the total number of cells in
the image, and $\lambda_{i1}$, the size of the cells from Process 1. Depending
on the model the process roughness $\alpha_{ir}$, $\nu_{ir}$ and prevalence
$\beta_{ik}$ parameters are either only weakly or not at all correlated with the
learned features. Interestingly, the convolutional network learns to represent
$\beta_{1}$ well, but not $\beta_{2}$ or $\beta_{3}$ -- this is consistent with
the fact that only $\beta_{1}$ influences the response $y_{i}$.

Finally, we observe that different models learn qualitatively different sets of
features. For example, the CNN features tend to be more clustered, with strong
correlation across several source features. In contrast, the RCF features show
more gradual shifts in correlation strength. They also show relatively little
variation in correlation strength across features other than $\lambda_{i1}$ and
$N_{i}$.

Note that the features do not naturally map onto one another from one run to the
next. This is not obvious from Figure \ref{fig:distributed_hm}, but a zoomed in
version in Figure \ref{fig:distributed_hm_subset} showing only the first 15
features per run makes this clear.

\begin{figure}
  \centering
  \includegraphics[width=\textwidth]{correlation_heatmap}
  \caption{Each feature learning algorithm (rows) learns a distributed
    representation of the true underlying features in the simulation. The two
    columns give results from two bootstrap resamples. Within each heatmap, rows
    correspond to the parameters in Table \ref{tab:sim_params}. Columns are
    activations of learned features; they have been reordered by a hierarchical
    clustering. The color of a cell gives the correlation between the associated
    true and learned features. Green and burgundy encode positive and negative
    correlation, respectively. Evidently, all learned features are associated
    with multiple true features at once.}
  \label{fig:distributed_hm}
\end{figure}

\subsection{Feature Learning vs. Inference}

It is natural to ask how the overall correlation between source and learned
features vary as a function of the algorithms and training sample sizes used. To
this end, in Figure \ref{fig:cca_sim}, we compute the top canonical correlations
between learned features.

Comparison in the CCA scores between the alternative approach

Comparison in feature stabilities across algorithms and sample sizes.

Comparison between the FDR's from the different selection strategies, at
different #samples. Idea is that selection diagram is rich, but not succinct
enough to draw meaningful comparisons across algorithms and training sample
sizes.

Specifically, plot the pi thr vs. number of features
selected. The point is that you want high $\pi$'s and some selections, because
those are unlikely to be false positives. You also don't want too many
selections overall at that given level, because that dilutes the signal.


\begin{figure}
  \centering
  \includegraphics[width=\textwidth]{combined_summary}
  \caption{}
  \label{fig:}
\end{figure}

\subsection{Stability study}

Figures \ref{fig:embeddings_sim} and \ref{fig:selections_sim} summarize the
results of Algorithms \ref{fig:fss} and \ref{alg:selection} applied to several
feature learners. For this figure, $\absarg{I} = \absarg{I^{C}} = 5000$ samples.
The analogous results with alternative splits are given in the appendix.

\begin{figure}
  \centering
  \includegraphics[width=\textwidth]{embeddings50}
  \caption{}
  \label{fig:}
\end{figure}

\begin{figure}
  \centering
  \includegraphics[width=\textwidth]{selection_paths50}
  \caption{}
  \label{fig:}
\end{figure}


Reading the embeddings figures
* Two axes are two dimensions of embeddings. We've chosen those that have high
selection scores
* Each star is a single sample, linked across bootstrap runs (there are 20 arms)
* Color is the true value of y
* Different panels show the different splits. First two are $I$, last is
$I^{C}$.

Interpretation about the embeddings -- difference in algorithms and
dimensionality reduction approaches?
* Different samples have different embedding stabilities. Seems like points near
the center are more stable, which makes sense -- the clouds of points along each
subspace are harder to align near the boundaries
* The clusters seem tigheter for SCA. More easily embeddable.
* CNN (and RCF?) features are more obviously associated with the response
* Top features in VAE are not necessarily associated with the response. Note
that this is in spite of the fact that the reconstructions are often quite good
(see appendix figure)

Interpretation about the selections -- difference in the algorithms and
dimensionality reduction approaches
* There is no obvious overfitting as far as feature learning is concerned.
* Selection paths $\Pi$ are not monotone for SCA. Likely due to the fact that
those scores can be correlated with one another in this approach, but not for
SVD.

\begin{figure}
  \centering
  \includegraphics[width=\textwidth]{}
  \caption{}
  \label{fig:}
\end{figure}
