
\section{Simulation Experiment}
\label{sec:simulation}

We use a simulation experiment to understand properties of the proposed
algorithms. Our guiding questions are,

\begin{enumerate}[label=(G{{\arabic*}})]
\item How different are the results obtained via supervised vs. unsupervised
  feature learning algorithms $T\left(\cdot; \theta\right)$?
\item When we vary the relative sizes of $I$ and $I^{C}$, we expect a trade-off
  between feature learning and inferential quality. Can we characterize this
  trade-off?
\item How does the dimensionality reduction approach used in $\mathcal{A}$
  affect downstream inferences?
\item Is it ever possible to substitute the RCF feature learner for either of
  the more time-consuming VAE or CNN models?
\end{enumerate}

To answer these questions, we evaluate our proposal using a full factorial with
three factors,

\begin{enumerate}
\item Relative sizes of $I$ and $I^{C}$: We use $\absarg{I} \in \left\{0.15n,
  0.5n, 0.9n\right\}$.
\item Dimensionality reduction procedure: We use both PCA and SCA.
\item Algorithm used: We train CNN, VAE and RCF models.
\end{enumerate}

We use $B = 20$ perturbations in each case. Hence, 180 models are trained, from
which 360 of dimensionality-reduced features are obtained.

\subsection{Simulated Data}

The key ingredient in this simulation is the construction of a dataset where all
``true'' features are directly controlled. To motivate the simulation, imagine
studying a set of tumor pathology slides, with the hope of discovering features
that are predictive disease prognosis. Each pathology slide gives a view into a
cellular ecosystem — there are several types of cells, with different sizes,
density, and affinities to one another.

In the simulation, we first generate these underlying properties of each slide.
Prognosis is defined as a linear function of these properties. Then, images are
generated that reflect these properties. The essential challenge is that the
investigator only has access to the images and patient prognoses, not the true
properties behind each image. A good feature extractor $T\left(x;
\hat{\theta}\right)$ should recover these important cell ecosystem properties
from the images alone. Example images for varying values of $y$ are given in
Figure \ref{fig:matern_example}. In total, our simulation generates 10,000 such
images, each of dimension $64 \times 64 \times 3$.

We now give details. The locations of cells are governed by an intensity
function drawn from a two-dimensional marked Log Cox Matern Process (LCMP)
\cite{diggle2013spatial}. Recall that a Matern process is a Gaussian process
with covariance function,
\begin{align}
  \label{eq:cov_lcmp}
C_{\nu, \alpha}(\|x - y\|)=\sigma^{2} \frac{2^{1-\nu}}{\Gamma(\nu)}\left(\sqrt{2 \nu} \frac{\|x - y\|}{\alpha}\right)^{\nu} K_{\nu}\left(\sqrt{2 \nu} \frac{\|x - y\|}{\alpha}\right),
\end{align}
where $\alpha$ acts like a bandwidth parameter and $\nu$ controls the roughness
of the simulated process.

Suppose we have $R$ types of cells. Then, our LCMP should have $R$ classes. This
can be constructed as follows. First, a nonnegative process
$\Lambda\left(x\right)$ is simulated along the image grid,
$\Lambda\left(x\right) \sim \exp{\mathcal{N}\left(0, \mathbf{C}_{\nu_{\Lambda},
    \alpha_{\Lambda}}\right)}$, where $\mathbf{C}_{\nu_{\Lambda},
  \alpha_{\Lambda}}$ is the covariance matrix induced by the $C_{\nu_{\Lambda},
  \alpha_{\Lambda}}$ in equation \ref{eq:cov_lcmp}. This is a baseline intensity
that determines the location of cells, regardless of cell type. Then, $R$
further processes are simulated, $B_{r}\left(x\right) \sim \exp{\beta_{r} +
  \mathcal{N}\left(0, \mathbf{C}_{\nu_{B}, \alpha_{B}}\right)} $. These
processes will reflect the relative frequencies of the $R$ classes at any
location $x$; the intercept $\beta_r$ makes a class either more or less frequent
across all positions $x$.

Given these intensity functions, we can simulate $N$ cell locations by drawing
from an inhomogeneous Poisson process with intensity $\Lambda\left(x\right)$.
For a cell at location $x$, we assign it cell type $r$ with probability
$\frac{B_{r}^{\tau}\left(x\right)}{\sum_{r^\prime = 1}^{R}
  B^{\tau}_{r^\prime}\left(x\right)}$. Here, we have introduced a temperature
parameter $\tau$ which controls the degree of mixedness between cell types at a
given location.

To complete the procedure for simulating images, we add two last source of
variation — the number of cells and the cell size. The number of cells per image
is drawn uniformly from 50 to 1000. The cells from class $R$ are drawn with a
random radius drawn from $\text{Gamma}\left(5, \lambda_{r}\right)$. A summary of
all parameters used to generate each image is given in Table
\ref{tab:sim_params}. These parameters are the ``true'' underlying features
associated with the simulated images; they give the most concise description of
the variation observed across the images.

\begin{figure}
  \centering
  \includegraphics[width=0.9\textwidth]{generation_mechanism}
  \caption{Example simulated images, for low (top), average (middle), and high
    (bottom) values of $y_i$. For each sample, three relative intensity
    functions $B_{k}\left(x\right)$ are generated, shown as a greyscale
    heatmaps. Samples drawn from each process are overlaid as circles. The final
    images, displayed at the right of each group, combines points from the three
    processes, removing the original generating intensity function, which is not
    available to the feature learner. Small $y_i$ values are associated with
    smoother, less structured intensity functions.}
  \label{fig:matern_example}
\end{figure}

\begin{table}[]
\begin{tabular}{|p{0.1\linewidth}|p{0.4\linewidth}|p{0.16\linewidth}|p{0.15\linewidth}|}
\hline
\textbf{Feature}              & \textbf{Description}                                 & \textbf{Influence}          & \textbf{Range}             \\
\hline
$N_i$                & The total number of cells.                                    & 0.5                         & $\left[50, 1000\right]$    \\
\hline
$\nu_{i,\Lambda}$    & The roughness of the overall intensity process.            & -0.5                        & $\left[0, 8\right]$        \\
\hline
$\alpha_{i,\Lambda}$ & The roughness of the overall intensity process.            & -0.5                        & $\left[0, 8\right]$        \\
\hline
$\beta_{ir}$             & The intercept controlling the frequency of class $r$. & 1 for $r = 1$,\ -1 otherwise & $\left[-0.15, 0.15\right]$ \\
\hline
$\nu_{iB}$           & The roughness of the relative intensity processes.            & -0.5                        & $\left[0, 3\right]$        \\
\hline
$\alpha_{iB}$        & The bandwidth of relative intensity processes.                & -0.5                        & $\left[0, 3\right]$        \\
\hline
$\tau_{i}$           & The temperature used in cell type assignment.                 & 0.5                         & $\left[0, 3\right]$        \\
\hline
$\lambda_{ir}$       & The shape parameter controlling the sizes of each cell type.  & 1 for $r = 1$,\ 0 otherwise  & $\left[100, 500\right]$   \\
\hline
\end{tabular}
\caption{Our simulation mechanism is governed by the above parameters.
  Parameters $N_i$ and $\lambda_{ir}$ control the number and sizes of imaginary
  cells. $\nu_{i, \Lambda}$, $\alpha_{i, \Lambda}$, $\beta_{ik}$, $\nu_{iB}$,
  $\alpha_{iB}$, and $\tau_{i}$ control the overall and relative intensities of
  the marked LCMP from which the cell positions are drawn. Example draws are
  displayed in Figure \ref{fig:matern_example}.}
\label{tab:sim_params}
\end{table}

These features are the underlying properties used to generate the prognosis for
each patient $i$. Specifically, we generate $y_i = \sum_{k} \text{Influence}_{k}
\times \left[\frac{\text{Feature}_{ik} -
    \overline{\text{Feature}}_{k}}{SD\left(\text{Feature}_{k}\right)}\right]$ for the
values Influence given in Table \ref{tab:sim_params}. Note that there is no
additional noise: if the $\text{Feature}_{ik}$'s were known for each sample,
then the $y_{i}$'s could be predicted perfectly. Therefore, the simulation
gauges the capacity of the feature learners to recover these known underlying
features.

\subsection{Results}

We now summarize findings of our factorial experiment. All simulated data,
trained models, and aligned features have been posted publicly; links can be
found in the appendix.

\subsubsection{Distributed features}

To answer G1, we directly visualize example learned features. Figure
\ref{fig:distributed_hm} shows the activations of learned features across 2000
images for two perturbed versions of the training data when $\absarg{I} = 0.9n$.
For the three algorithms, the learned features correspond to,

\begin{itemize}
\item CNN: Activations from the last layer of neurons, used directly as input
  for the regression. There are a total of 512 nonnegative
  features\footnote{They are nonnegative because they follow an ReLU layer.}.
\item VAE: Spatially-pooled activations from the middle, encoding layer of the
  variational autoencoder. There are a total of 64 real-valued features.
\item RCF: The spatially-pooled activations corresponding to each of 1048 random
  convolutional features.
\end{itemize}

Our first observation is that, across algorithms, there is no simple
correspondence between learned and source features (i.e., parameters of the
underlying simulation). For example, it is not the case that one set of features
represents the number of cells $N$, and a different set maps to the roughness
$\nu_{\Lambda}$. Rather, there appear to be clusters of learned features, and
each cluster corresponds to a pattern across multiple source features. We also
find large subsets of features, across all models, that are only weakly
correlated with any source feature.

Next, note that certain source features are ``easier'' to represent than others,
in the sense that more of the learned features are strongly correlated with
them. Many features are correlated with $N_{i}$, the total number of cells in
the image, and $\lambda_{i1}$, the size of the cells from Process 1. Depending
on the model, the bandwidth / roughness $\alpha_{ir}$, $\nu_{ir}$ and prevalence
$\beta_{ik}$ parameters are either only weakly or not at all correlated with the
learned features. Interestingly, the convolutional network learns to represent
$\beta_{1}$ well, but not $\beta_{2}$ or $\beta_{3}$ -- this is consistent with
the fact that only $\beta_{1}$ influences the response $y_{i}$. Even when
features learn to detect variation in $\alpha_{ir}$ and $\nu_{ir}$, they cannot
disambiguate between these two parameters.

Finally, consider differences between feature learning algorithms. The CNN and
VAE features tend to be more clustered, with strong correlation across several
source features. In contrast, the RCF features show more gradual shifts in
correlation strength. They also show relatively little variation in correlation
strength across features other than $\lambda_{i1}$ and $N_{i}$.

Note that the features do not naturally map onto one another from one run to the
next. This is not obvious from Figure \ref{fig:distributed_hm}, but a zoomed in
version in Supplementary Figure \ref{fig:distributed_hm_subset} showing only the
first 15 features per run makes this clear.

\begin{figure}
  \centering
  \includegraphics[width=\textwidth]{correlation_heatmap}
  \caption{Each feature learning algorithm learns a distributed
    representation of the true underlying features in the simulation. The two
    columns give results from two runs. Within each heatmap, rows correspond to
    the parameters in Table \ref{tab:sim_params}. Columns are activations of
    learned features; they have been reordered by a hierarchical clustering
    using the package \citep{barter2018superheat}. The color of a cell gives the
    correlation between the associated true and learned features. Green and
    burgundy encode positive and negative correlation, respectively. All learned
    features are associated with multiple true features at once.}
  \label{fig:distributed_hm}
\end{figure}

\subsubsection{Feature Learning vs. Inference}

To answer G2 and G3, Figure \ref{fig:cca_sim} displays the top canonical
correlations between learned features $\underaccent{\bar}{\mathbf{Z}}_{b}$ and
the original source features, across bootstrap replicates and algorithms. Note
that we calculate these scores separately for training, development, and testing
splits. The training and development splits are subsets of $I$. Training samples
were used in the optimization for each feature learner; development samples were
used to choose hyperparameters of the optimizer. They are shown separate from
the test samples $I^{C}$ to make it possible to recognize potential overfitting
in the feature learning process.

Even after the initial dimensionality reduction, only several CCA directions
have high canonical correlations. The dimensionality reduction method used does
not have much effect at this stage. In general, the fraction of data belonging
to $I$ also matters little; however, there is an exception in the CNN features.
Here, the features learned when $\absarg{I}$ is only 15\% of the data have
noticeably lower canonical correlation in the second dimension. Note also that,
from the point of view of these canonical correlations, the RCF and VAE have
comparable feature learning behaviors.

The feature learning algorithms do not seem to overfit the simulation data. If
they did, then the canonical correlations on the training and development data
would be larger than those on the test data. That said, there are noticeable
differences between the data splits, and this effect will be visible in later
figures as well. We hypothesize that the differences are due to the alignment
process. During the Procrustes rotation, one matrix
$\mathbf{Z}_{b}\left[I^{C}\right]$ may get ``lucky'' and learn an alignment with
axes that better reflect variation in the source features. In this way, even
though the test data may not have been used to learn features, they may have
higher correlation with the source features than the original
$\mathbf{Z}_{b}\left[I\right]$.

\begin{figure}
  \centering
  \includegraphics[width=0.8\textwidth]{cca_summary}
  \caption{The top canonical correlations between the sources features and the
    aligned $\protect\underaccent{\bar}{\mathbf{Z}}_{b}$. Horizontal rows
    correspond to CNN, RCF, and VAE algorithms. Columns give the CCA dimension.
    Within each display, the canonical correlations across all bootstraps is
    shown. The $x$-axis within each cell gives the amount of data used for
    feature learning and the dimensionality reduction method used. Canonical
    correlations are computed separately for samples within training,
    development, and test splits, indicated by each point's color.}
  \label{fig:cca_summary}
\end{figure}

To shed further light on G2 and G3, Figure \ref{fig:selection_summary} shows
the median number of features selected by stability selection
$\absarg{S\left(\lambda, \pi_{thr}\right)}$ across training sample sizes. The
median is taken across all bootstrap iterations. We fix a regularization
strength $\lambda$ where small perturbations in $\pi_{thr}$ lead to large
changes in the number of selected features. We have run stability selection for
$250$ replicates, all restricted to either training, development, or test data,
as indicated by the color of each line.

All the curves decrease because increasing the stringency $\pi_{thr}$ leads to
fewer selected features. We find that the features learned by the CNN are more
frequently selected. This is expected, considering that the CNN features are
learned in a supervised fashion. More surprising is the fact that the RCF
features are more frequently selected than the VAE features, suggesting that the
simple baseline might already provide features useful for interpretation, giving
an affirmative answer to G4.

Finally, it appears that $\absarg{S\left(\lambda, \pi_{thr}\right)}$ is largest
when using 50\% of the data for feature learning. Though the features learned
using 90\% of the data may be higher quality, it is not possible to select them
as frequently, because stability selection will have low power when it can only
subsample from the 10\% of the data reserved for inference. This phenomenon
appears for features learned by both supervised and unsupervised methods. For
this reason, in the remainder of this study, we will focus on results obtained
using a 50\% split between learning and inference -- that is, $\absarg{I} =
\absarg{I^{C}}$ -- though results using different splits are available in the
appendix.

\begin{figure}
  \centering
  \includegraphics[width=\textwidth]{selection_summary2}
  \caption{The median number of features selected by stability selection across
    bootstrap runs $\protect\underaccent{\bar}{\mathbf{Z}}_{b}$, viewed as a
    function of $\pi_{thr}$. Rows and columns correspond to models and fraction
    of the data used for training, respectively. Stability selection is run
    within training, validation, and test splits (color) and using two
    dimensionality reduction procedures during alignment (solid vs. dashed
    lines).}
  \label{fig:selection_summary}
\end{figure}

\subsubsection{Stability visualization}

Figures \ref{fig:sim_embeddings-pca-1-2} and \ref{fig:selection_paths-5-2}
summarize the results of Algorithms \ref{alg:fss} and \ref{alg:selection}
applied to several feature learners. Within a single panel, each star glyph
corresponds to a single sample. The arms of the glyph link the bootstrap
replicates for that given sample: $\protect z_{i}^{1}, \dots, z_{i}^{20}$. Each
glyph is colored in the true value of $y_{i}$ for that sample. Only the first
two aligned feature dimensions are shown.

Both the model and reduction strategy used influence the stability of the
learned features. Based on the relative sizes of the glphys, the CNN features
are least stable, those from the RCF are most stable, and the VAE features are
intermediate. Separate regions of the learned feature space appear more stable
than others. Features learned with 15\% of the data are highly unstable and show
little association with the response. For larger $\absarg{I}$'s, the learned
features are more stable. Supplementary Figure
\ref{fig:sim_embeddings-full-pca-1-2} shows that in this case, there are also
stronger associations with $\mathbf{y}$.

\begin{figure}
  \centering
  \includegraphics[width=\textwidth]{sim_embeddings--pca-1-2}
  \caption{Example aligned features $\protect\underaccent{\bar}{\mathbf{Z}}_{b}$
    derived from several models with varying feature learning sample sizes
    $\absarg{I}$. For clarity, only a subset of samples is shown. For a version
    of this figure showing all samples, and collapsing each star to a point,
    refer to Supplementary Figure \ref{fig:sim_embeddings-full-pca-1-2}.}
  \label{fig:sim_embeddings-pca-1-2}
\end{figure}

Figure \ref{fig:selection_paths-5-2} displays stability selection paths
$\Pi_{k}^{b}\left(\lambda\right)$ for combinations of feature learning and
dimensionality reduction procedures. Like in Figure \ref{fig:cca_summary}, we
find that the largest variation is between the data splits, but that
overfitting does not appear to be a problem. Indeed, for several features, the
development and test splits have higher selection probabilities.

Revisiting G3, we find the first major differences between dimensionality
reduction strategies. The selection paths are not monotone for SCA, and they
also vary substantially across runs. This is most likely a result of the fact
that the matrix $B$ in the decomposition $X = Z B Y^{T}$ need not be diagonal,
which correlates the resulting coordinates. Further, when using the PCA, the top
features are also the most selected ones; this is not the always the case with
SCA.

Next, consider G1 and G4, from the selective perspective. Between algorithms, we
find that the selection paths for RCF features form a tighter band across all
bootstrap samples, another mark in its favor. For all algorithms, this band
seems to widen for later feature dimensions; the selection probabilities also
generally rise more gradually. For Feature 3 onwards, the selection paths for
the VAE rise more gradually than the corresponding paths for either the CNN or
RCF dimensions, suggesting a lack of association with $\mathbf{y}$.

\begin{figure}
  \centering
    \makebox[\textwidth][c]{\includegraphics[width=1.3\textwidth]{selection_paths-5-2}}
  \caption{Stability selection paths $\hat{\Pi}_{k}^{b}\left(\lambda\right)$ for
    10 learned features, across models and alignment strategies. Top and bottom
    groups use PCA and SCA-reduced features, respectively, and each row
    corresponds to a feature learning algorithm. Each column corresponds to one
    learned feature dimension. Each curve gives
    $\hat{\Pi}_{k}^{b}\left(\lambda\right)$ for one bootstrap replicate $b$
    restricted to samples from either the training, development, or test set.
    The $x$-axis is the regularization strength $\log \lambda$ and the $y$-axis
    is the probability of selecting that feature at the given regularization
    strength.}
  \label{fig:selection_paths-5-2}
\end{figure}
