
\section{Data Analysis}
\label{sec:dataset}

In this section, we conduct a feature stability analysis on the spatial
proteomics dataset\footnote{The data are
  \href{https://www.angelolab.com/mibi-data}{publicly available}. See also the
  appendix for preprocessed versions.} reported in the study
\citep{keren2018structured}, which found a relationship between the spatial
organization of Triple Negative Breast Cancer (TNBC) tissues and disease
progression. In a classical proteomics study, the expression levels for a set of
proteins is measured for a collection of cells, but the cell locations are
unknown. In contrast, these data provide for each patient (1) an image
delineating cell boundaries and (2) the protein expression levels associated
with each cell in the images.

We will work only with the spatial cell delineations, not the protein expression
levels. This allows us to study the mechanics of feature learning within the
images without having to worry about linking the expression and image data,
which is in itself a complex integration problem. Though this means we lose some
scientific depth, we gain substantial implementation simplicity, and the
analysis serves as a clear illustration. Our complete data are 41 $2048 \times
2048$-dimensional images, each taken from a separate patient. We associate each
pixel with one of 7 categories of tumor and immune cell types.

\begin{figure}
  \centering
  \includegraphics[width=\textwidth]{example_cells}
  \caption{Example patches from the TNBC data. Panels are ordered by $y_i$, the
    (log) fraction of cells they contain that belong to the tumor. The goal of
    the prediction algorithm in this example is to correctly place patches from
    new patients along this gradient, after having observed many patches and
    their corresponding $y_i$'s.}
  \label{fig:example_cells}
\end{figure}

To setup a prediction problem, we first split each image into $512 \times 512 \times 7$
patches. These patches are our $x_{i}$. Patches from 32 of the patients are
reserved form feature learning. Four among these 32 are used as a development
split, to tune parameters of the feature learning algorithms. As a response
variable, we use $y_{i} = \log\left(\frac{\#\{\text{Tumor cells in
  }x_{i}\}}{\#\{\text{Immune cells in }x_i\}}\right)$. Example cell patches are
shown in Figure \ref{fig:example_cells}.

As a baseline, we compare against a ridge regression with pixelwise composition
features. Specifically, we train a model with $\mathbf{y}$ as a response and the
average number of pixels belonging to each of the cell-type categories as a
(length 7) feature vector. This helps us to determine whether the model has
learned more interesting features useful for counting cells, like cell size and
boundaries, rather than simply averaging across pixel values. Indeed, Figure
\ref{fig:tnbc_baseline} makes clear that, with the exception of the RCF-SCA
combination, all feature learning - dimensionality reduction combinations
outperform this manual baseline.

\begin{figure}
  \centering
  \includegraphics[width=0.8\textwidth]{tnbc_baseline}
  \caption{Relative performance of feature learning strategies on the TNBC data.
    Linked points come from the same bootstrap replicate. The manual features
    refers to the regression model that simply uses the raw pixel counts for
    each of the cell types. Predictions from the development split are omitted;
    this split has few patches, and the estimated MSE's have high variance.}
  \label{fig:tnbc_baseline}
\end{figure}

Stability curves associated with the learned features from the CNN, RCF, and VAE
models are shown in Figure \ref{fig:tnbc_selection_paths-2}. Interestingly,
across all algorithms, the top aligned features are not necessarily those with
the highest selection probabilities. For example, CNN Feature 4 has higher
selections than Feature 3, RCF Feature 6 has higher selection than Feature 4,
and VAE Feature 5 is more frequently selected than Feature 3. In contrast to the
simulation, the CNN and VAE do not have substantial differences in their
selection curves. Further, the RCF features seem to have high selection
probabilities across more dimensions than either the CNN or VAE. This suggests
that the most salient features in $x_i$ are also relevant for predicting $y_i$,
and that supervision is not as critical in this problem as it was in the
simulation.

\begin{figure}
  \centering
    \makebox[\textwidth][c]{\includegraphics[width=1.3\textwidth]{tnbc_selection_paths-2}}
  \caption{Selection curves for features learned from the TNBC data. 50\% of
    the data were used for feature learning and PCA was used for
    dimensionality reduction. Each curve is read as in Figure
    \ref{fig:selection_paths-5-2}. The analogous paths using SCA alignment are
    given in Supplementary Figure \ref{fig:tnbc_selection_paths-1}.}
  \label{fig:tnbc_selection_paths-2}
\end{figure}

Example aligned coordinates are given in Figure
\ref{fig:tnbc_embeddings-1-2}. Consistent with the conclusion, we find that
the association with the response is clearly visible with respect to the learned
feature dimensions, even when using unsupervised algorithms. The changes in the
sizes of glyphs across regions of the learned feature space are especially
pronounced in this application. For example, in the VAE, the representations of
samples with higher tumor-to-immune ratio $y_i$ are much more stable than those
with low ratio.

\begin{figure}
  \centering
  \includegraphics[width=\textwidth]{tnbc_embeddings-1-2}
  \caption{Representation of samples according to some of the important learned
    feature dimensions. Glyphs are read as in Figure
    \ref{fig:sim_embeddings-pca-1-2}. To interpret these features, example
    patches are arranged according to these coordinates in Figure
    \ref{fig:tnbc_imagegrid-PCA-1-2} and Supplementary Figure
    \ref{fig:tnbc_imagegrid-SCA-1-2}. Only a subsample of points are shown; the
    full dataset with stars collapsed to points is shown in Figure
    \ref{fig:tnbc_embeddings-full-both-1-2}.}
  \label{fig:tnbc_embeddings-1-2}
\end{figure}

There is no consensus on how to best interpret automatically learned features
\citep{doshi2017towards}. Nonetheless, we present one simple approach in Figure
\ref{fig:tnbc_imagegrid-PCA-1-2}, overlaying example patches onto aligned
coordinates. For example, in the CNN, the first dimension distinguishes between
the relative number of tumor and immune cells, while the second dimension
reflects the density of cells. In the RCF, the second dimension captures the
diversity of cell types, with more uniform samples on the left and more
heterogeneous ones on the right. The second dimension of the VAE seems related
to both cell density and number of cell types. The analogous display from an SCA
reduction is given in Supplementary Figure \ref{fig:tnbc_imagegrid-SCA-1-2}.

\begin{figure}
  \centering
  \makebox[\textwidth][c]{\includegraphics[width=1.2\textwidth]{tnbc_imagegrid-PCA-1-2}}
  \caption{A version of the PCA panels of Figure \ref{fig:tnbc_embeddings-1-2}
    that overlays representative samples across the learned feature space. Cells
    are color coded as in Figure \ref{fig:example_cells}. Note that the overall
    shape of the region in which images are displayed mirrors the shape of the
    cloud of points in Figure \ref{fig:tnbc_embeddings-1-2}.}
  \label{fig:tnbc_imagegrid-PCA-1-2}
\end{figure}
