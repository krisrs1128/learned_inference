
\section{Reproducibility}

Our raw and derived data are available at the following links.

Instructions to reproduce our simulations and data analysis example are
available in a \href{https://github.com/krisrs1128/learned_inference}{README} on
the study's github page. Data preprocessing, the regression baseline, and
feature learning can be reproduced in the ipython notebooks

\begin{itemize}
\item \texttt{tnbc\_splits.ipynb}
\item \texttt{tnbc\_baseline.ipynb}
\item \texttt{model\_training.ipynb}
\end{itemize}

Data simulation, feature stability analysis, and visualization of results are
done within the rmarkdown documents,

\begin{itemize}
\item \texttt{generate.Rmd}
\item \texttt{stability.Rmd}
\end{itemize}

To support code reusability between experiments, two helper packages were prepared,

\begin{itemize}
\item \texttt{stability}
\item \texttt{inference}
\end{itemize}

We have prepared a \href{https://hub.docker.com/r/krisrs1128/li}{docker image}
with all necessary software installed. For example, to reproduce figure
\ref{fig:?}, you can enter the image and execute the relevant rmarkdown document
using the following commands,

\begin{verbatim}
shell> docker run -it krisrs1128/li:latest bash
docker shell> git pull krisrs1128/learned_inference.git
docker shell> source learned_inference/.env
docker shell> # download relevant data
docker shell> Rscript rmarkdown -e "rmarkdown::render('learned_inference/inference/vignettes/stability.Rmd')"
\end{verbatim}

\section{Model Outputs}

\begin{figure}
\includegraphics[width=\textwidth]{reconstructions}
\caption{}
\label{fig:reconstructions}
\end{figure}
