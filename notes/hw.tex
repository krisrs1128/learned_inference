\documentclass[]{article}
\usepackage{url}
\usepackage{amssymb, amsfonts, amsmath, amsthm}
\usepackage{graphicx, natbib}
\usepackage[colorlinks=true]{hyperref}
\usepackage[ruled,vlined]{algorithm2e}
\usepackage{accents}
\usepackage{enumitem}
\usepackage{authblk}

% ------------------------------------------------------------------------
% Macros
% ------------------------------------------------------------------------
%~~~~~~~~~~~~~~~
% List shorthand
%~~~~~~~~~~~~~~~
\newcommand{\BIT}{\begin{itemize}}
\newcommand{\EIT}{\end{itemize}}
\newcommand{\BNUM}{\begin{enumerate}}
\newcommand{\ENUM}{\end{enumerate}}
%~~~~~~~~~~~~~~~
% Text with quads around it
%~~~~~~~~~~~~~~~
\newcommand{\qtext}[1]{\quad\text{#1}\quad}
%~~~~~~~~~~~~~~~
% Shorthand for math formatting
%~~~~~~~~~~~~~~~
\newcommand\mbb[1]{\mathbb{#1}}
\newcommand\mbf[1]{\mathbf{#1}}
\def\mc#1{\mathcal{#1}}
\def\mrm#1{\mathrm{#1}}
%~~~~~~~~~~~~~~~
% Common sets
%~~~~~~~~~~~~~~~
\def\reals{\mathbb{R}} % Real number symbol
\def\integers{\mathbb{Z}} % Integer symbol
\def\rationals{\mathbb{Q}} % Rational numbers
\def\naturals{\mathbb{N}} % Natural numbers
\def\complex{\mathbb{C}} % Complex numbers
\def\simplex{\mathcal{S}} % Simplex
%~~~~~~~~~~~~~~~
% Common functions
%~~~~~~~~~~~~~~~
\renewcommand{\exp}[1]{\operatorname{exp}\left(#1\right)} % Exponential
\def\indic#1{\mbb{I}\left({#1}\right)} % Indicator function
\providecommand{\argmax}{\mathop\mathrm{arg max}} % Defining math symbols
\providecommand{\argmin}{\mathop\mathrm{arg min}}
\providecommand{\arccos}{\mathop\mathrm{arccos}}
\providecommand{\asinh}{\mathop\mathrm{asinh}}
\providecommand{\dom}{\mathop\mathrm{dom}} % Domain
\providecommand{\range}{\mathop\mathrm{range}} % Range
\providecommand{\diag}{\mathop\mathrm{diag}}
\providecommand{\tr}{\mathop\mathrm{tr}}
\providecommand{\abs}{\mathop\mathrm{abs}}
\providecommand{\card}{\mathop\mathrm{card}}
\providecommand{\sign}{\mathop\mathrm{sign}}
\def\rank#1{\mathrm{rank}({#1})}
\def\supp#1{\mathrm{supp}({#1})}
%~~~~~~~~~~~~~~~
% Common probability symbols
%~~~~~~~~~~~~~~~
\def\E{\mathbb{E}} % Expectation symbol
\def\Earg#1{\E\left[{#1}\right]}
\def\Eabsarg#1{\E\left|{#1}\right|}
\def\Esubarg#1#2{\E_{#1}\left[{#2}\right]}
\def\P{\mathbb{P}} % Probability symbol
\def\Parg#1{\P\left({#1}\right)}
\def\Psubarg#1#2{\P_{#1}\left[{#2}\right]}
\def\Cov{\mrm{Cov}} % Covariance symbol
\def\Covarg#1{\Cov\left[{#1}\right]}
\def\Covsubarg#1#2{\Cov_{#1}\left[{#2}\right]}
\def\Var{\mrm{Var}}
\def\Vararg#1{\Var\left(#1\right)}
\def\Varsubarg#1#2{\Var_{#1}\left(#2\right)}
\newcommand{\family}{\mathcal{P}} % probability family
\newcommand{\eps}{\epsilon}
\def\absarg#1{\left|#1\right|}
\def\msarg#1{\left(#1\right)^{2}}
\def\logarg#1{\log\left(#1\right)}
%~~~~~~~~~~~~~~~
% Distributions
%~~~~~~~~~~~~~~~
\def\Gsn{\mathcal{N}}
\def\Ber{\textnormal{Ber}}
\def\Bin{\textnormal{Bin}}
\def\Unif{\textnormal{Unif}}
\def\Mult{\textnormal{Mult}}
\def\Cat{\textnormal{Cat}}
\def\Gam{\textnormal{Gam}}
\def\InvGam{\textnormal{InvGam}}
\def\NegMult{\textnormal{NegMult}}
\def\Dir{\textnormal{Dir}}
\def\Lap{\textnormal{Laplace}}
\def\Bet{\textnormal{Beta}}
\def\Poi{\textnormal{Poi}}
\def\HypGeo{\textnormal{HypGeo}}
\def\GEM{\textnormal{GEM}}
\def\BP{\textnormal{BP}}
\def\DP{\textnormal{DP}}
\def\BeP{\textnormal{BeP}}
%~~~~~~~~~~~~~~~
% Theorem-like environments
%~~~~~~~~~~~~~~~
\theoremstyle{definition}
\newtheorem{definition}{Definition}[section]

%-----------------------
% Probability sets
%-----------------------
\newcommand{\X}{\mathcal{X}}
\newcommand{\Y}{\mathcal{Y}}
\newcommand{\D}{\mathcal{D}}
\newcommand{\Scal}{\mathcal{S}}
%-----------------------
% vector notation
%-----------------------
\newcommand{\bx}{\mathbf{x}}
\newcommand{\by}{\mathbf{y}}
\newcommand{\bt}{\mathbf{t}}
\newcommand{\xbar}{\overline{x}}
\newcommand{\Xbar}{\overline{X}}
\newcommand{\tolaw}{\xrightarrow{\mathcal{L}}}
\newcommand{\toprob}{\xrightarrow{\mathbb{P}}}
\newcommand{\laweq}{\overset{\mathcal{L}}{=}}
\newcommand{\F}{\mathcal{F}}


\begin{document}

\begin{enumerate}

\item Write down a simplified model of feature selection when features have to
be learned. You do not need to explain how inference is conducted, but you
should fully specify the generative mechanism behind an observed dataset
(covariates and response). How does this model differ from approaches where the
features are not considered to be learned? How does it compare and contrast to a
standard errors-in-variables model?

To model the situation where features must be learned before performing
regression, consider,
\begin{align*}
Z &= LF^T + E & F &\sim \Gsn\left(0, \diag\left(\*\sigma_{F}^2\right)\right)\\
y &= L \beta + \epsilon & E &\sim \Gsn\left(0, \sigma_{E}^2 I_{D}\right)\\
& & \epsilon &\sim \Gsn\left(0, \sigma_{\epsilon}^2\right)
\end{align*}
where $L \in \reals^{N \times K}, F \in \reals^{D \times K}, Z \in \reals^{N
\times D}, \beta \in \reals^{K}$. We assume that $\beta$ has only $S$ nonzero
entries. We require that only $Z$ — not $L$ — is available for predicting the
response $y$. $Z$ is interpreted as a set of features algorithmically derived
from some collection $x_1, \dots, x_N$ (for example, a set of images or
documents). $L$ contains the true, but inaccessible, features that generates
$y$. The unequal variance across columns of $F$ reflects that some true features
are more easily extracted than others. Since a feature extractor can be applied
multiple times, we extend this to
\begin{align*}
Z^{b} &= \left(L\Pi^{b}\right)\left(F \Pi^{b}\right)^T + E^{b} & E^{b} &\sim \Gsn\left(0, \diag\left(\*\sigma_{E}^2\right)\right) \\
y &= L \beta + \epsilon & \epsilon &\sim \Gsn\left(0, \sigma_{\epsilon}^2\right)
\end{align*}
where $b = 1, \dots, B$ and $\Pi \in \reals^{K \times K}$ is a permutation
matrix, reflecting the fact that columns of $Z$ need not correspond across the
$B$ runs. The inaccessibility of $L$ distinguishes this approach from standard
regression. This model is related to the errors-in-variables in model, except
(1) the covariates are effectively low-dimensional, and (2) $B$
(column-permuted) versions of the same covariates can be extracted.

\item Discuss alternative learning inferential approaches for the model
discussed in (1). What choices have to be made w.r.t. supervision in feature
learning, dimensionality of the learned features, and evaluation of learned
feature significance?

The analyst’s goal is to recover $L$ and, given that, $\beta$. We will discuss
feasibility below. First, however, consider choices that must be made when
pursuing this goal,
\begin{itemize}
\item Feature Extractor: A feature extractor is a mapping $\mathcal{Z}_{\theta}:
\left(x, y\right) \to z$. ``Learning’’ is the process of choosing $\theta \in
\Theta$. The family $\Theta$ of potential extractors must be chosen. They may
differ in the dimensionality $K$ of the learned $z_i$ for example.
\item Supervision: If $\mathcal{Z}_{\theta}$ makes use of the response $y$, then
is called supervised. If it only requires $x$, it is unsupervised.
\item Data splitting: Should all the data used for both learning $\theta$ and
estimating $\beta$? If not, how should the samples be split? We will see that
data splitting is useful in the supervised setting.
\end{itemize}

\end{enumerate}
\end{document}
